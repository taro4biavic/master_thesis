\documentclass{article}
\usepackage[utf8]{inputenc}
\usepackage{amsfonts}
\usepackage{mathrsfs}
\usepackage[ruled, lined, linesnumbered, commentsnumbered, longend]{algorithm2e}
\title{Edge Bundling and Visualization of Dynamic Graph}
\author{Hao-Yu Zheng }
\date{}

\begin{document}

\maketitle
\begin{abstract}
    
\end{abstract}

\section{Introduction}
Data visualization is now a trending technique since massive data are accessible due to the fact that internet and computers are easily available, people nowadays yield far more data compared to those in decades ago. \emph{Edge Bundling} is one of a favored option of data visualization. In graph drawing problem on node-link diagram, where edges and vertices represent the relations between different entities. Clutters would be produced if the amount of data are proportionally larger than the area to display\cite{TAXONOMY}, edge bundling technique provide visual simplification without losing structural meaning of adapted graph, giving users better insight and comprehension to the relations between nodes.
\par 


\section{Related Work}
\subsection{Dynamic Graph}
The world we human to observe is nothing but a dynamic world, that is to say making graph-structure models to depict our observation with dynamic-view is intuitive.

\begin{equation}
e \in E = \{n_{start}(e) \in V, n_{end}(e) \in V, t_{start} \in \mathbb{R}^+, t_{end}\in\mathbb{R}^+\}
\end{equation}


One of definition of dynamic graph is characterized in Eqn. 1 \cite{BUNDLED_DG} . A Dynamic graph $G = (V,E)$ is a graph with vertices set $V$ and edges set $E$, each edges $e \in E$ has two end-points $n_{start}(e)$ and $n_{end}(e)$, and with the existence of temporal tag $t$, a lifetime of an edge can be defined as $[t_{start},t_{end},t_{end}>t_{start}]$. A primary goal of data Visualization of dynamic graph is to preserve the mental-map of users\cite{SOTA_DG}, which means we desire no confusion is generated when time goes forward, especially under the condition that the graph vary significantly between the interval of unit time. There are some characteristics also in our expectations to satisfy, such as readability, scalability and efficiency. 


\subsection{Edge Bundling}
In this section, we would dive into the topic thoroughly about the technique of edge bundling, which is our emphasis in this thesis. Node-link diagram has long be a pragmatic representation of data, compared to other representation method of data such as matrix-view or parallel coordinates plot, node-link diagram is robust especially on the dataset that relation matters, because of the neat metaphor for relations between entities by nodes and links. However, the clutter issue would be a problem if the amount of elements become heavy, the resulting sophisticated graph would make viewer exhausted or confused in understanding. 
\par
To alleviate this problem, many visualization techniques were proposed, edge bundling is one of a kind among them, one of the most early and famous work is Hierarchical Edge Bundling(HEB)\cite{HEB} proposed by Hotlen, the work applied B-spline\cite{BSpline} model to bundle adjacency edges, make it an intuitive and continuous way to reduce visual clutter. Nonetheless, HEB can only be executed on tree-like structure graph. Therefore, aims at getting a generic solution, Cui \emph{et al.} proposed Geometry-Based Edge Bundling(GBEB)\cite{GBEB}. GBEB introduced the first notable geometry-based method which gather edges by control mesh, and the mesh is generated by underlying graph structure. GBEB has some drawbacks, for instance, the effectiveness of the system is dominated by the quality of control mesh hence is unstable in execution time, and the semantic meaning of outcome might be weak because of the algorithm, despite these shortcomings, the work offered a novel solution to edge bundling in general graph.Following the idea, many works came with similar concepts to cope with the problem\cite{WindingRoad,EBEB,GEOREF,CBEB}. Another noteworthy resolution is Force-Directed Edge Bundling(FDEB)\cite{FDEB}, proposed by Hotlen \emph{et al}., using physical simulation method to reduce clutter by calculating the spring force between interacting edges, the advantage of this work is that it is easy to follow without complex concept, in contrast to GBEB which has multiple steps and techniques, there are plenty of works were presented after the idea of FDEB\cite{3DFDEB,TGIEB,StreamEB}, they either modify or extend based on this work. 
\par 
The above mentioned works built the solid fundamental for edge bundling, several studies were addressed based on these works and got decent results, despite this fact, imaged-based solution\cite{Skeleton,FFTEB,Smooth} came as a new branch with trait of easily and effectively implemented. Kernel Density Estimation Edge Bundling(KDEEB)\cite{KDEEB} was the pioneered representative of this kind, KDEEB utilizes kernel density estimation to generate a density map given input graph, then the sampled edges shift to the direction of local density maxima and hence form bundles. FFTEB\cite{FFTEB} replace image space in the bundling process with spectral space, therefore increasing computational speed. Hurter, \emph{et al.} applied KDEEB on dynamic graph \cite{Smooth}, which is the inspiration for us.
\par
The metrics to measure the quality of edge bundling is of course an important factor for us to consider. Readability, which is an essential evaluation to measure if a graph drawing algorithm is good, many research had been went through to introduce aesthetic standard in terms of readability\cite{Met1,Met2,Met3}, Helen found that elimination of edge crossings could be profitable for people in understanding of graphs, and she also discovered that aligning nodes and edges to an underlying grid is crucial as well with user study. Fortunately, edge bundling, as we would elaborate in following section, possess the ability of eliminating edge crossings by its nature. Some studies proposed metrics especially for edge bundling methods. \cite{EMet1} Wu, *et al.* quantify the information sent after edge bundling by entropy. \cite{FAITH} Nguyen *et al.* takes the perspective of faithfulness, a graph bundling is said to be faithful if its representation is logically consistent with the data 
it behaved. \cite{VALID} Saga conducted a questionnaire task to summarize the quantitative measures that are strongly bonded with human cognition, which are the metrics to measure readability. Above studies give us a good insight to build a appropriate metric for our problem, when it comes to evaluating dynamic graph edge bundling, we would prioritize preserving mental map; therefore the pick of metric for our work would based on this criteria.
\section{Background}
In this section, 
\subsection{Fast Fourier Transform Edge Bundling}
In this part, we would elaborate the details of \cite{FFTEB}Fast Fourier Transform Edge Bundling that is evidently related to our work, including but not limited to the problem model, the inspiration of this work and the process.
\subsubsection{Problem Model}
We've been introduced multiple methods of edge bundling so far, but have yet to give the definition. Given $G = (V,E)$ be a graph with node set $V = \{v_i\}$ and edge set $E = \{e_i\}$. Let $D:E\to\mathbb{R}^2$ be a drawing operator, which project edges $e\in E$ into a graph drawing form, $D(e)$ may refer to a straight line or other types of line segment. Let $B:D(G)\to B(G)$ be a bundling operator of a graph. $B$ is considered a edge bundling algorithm if
\begin{displaymath}
\forall (e_i\in G, e_j\in G)|\kappa(e_i,e_j)<\kappa_{max}\to
\end{displaymath}
\begin{equation}
\delta(B(D(e_i)),B(D(e_j)))\ll \delta(D(e_i),D(e_j))
\end{equation}

The Eqn 2. indicates any edges in graph that have higher similarity than $\kappa_{max}$ would be bundled, and the results gain more spatially similar in contrast to those unbundled ones. $\kappa:G\times G \to \mathbb{R}^+$ is the compatibility function, which is to measure if a pair of edge should be bundled or not, the higher the value is, the more dissimilar they are, $\kappa_{max}$ here represents the least degree of similarity that is acceptable for users. $\delta$ is the distance function to quantify the closeness between two graph drawings.

\subsubsection{Kernel Density Estimation Edge Bundling}
Kernel Density Estimation Edge Bundling(KDEEB) is an image-based method that apply the concept of mean-shift\cite{MEANSH} to process bundling, mean-shift is the process of sample points find their best way to approach local density maximum, the term best here means the way density grown fastest. And the density in our case is corresponded to edge clutter in graph. To calculate local maximum, defines model $\rho:\mathbb{R^2}\to \mathbb{R^+}$ 
\begin{equation}
\rho(x) = \sum^{N}_{i=1}\int_{y\in e_i}K(\frac{x-y}{h})
\end{equation}
The kernel function $K$ is used to measure the similarity of edges, Epanechnikov function is a quadratic form kernel that is commonly chosen because of its efficiency. Density map can be obtained by convolving $K$ around the graph, and then we can use the results to calculate the gradients with following equation:

\begin{equation}
\frac{dx}{dt} = \frac{h(t)\nabla\rho(t)}{max(\Vert\nabla\rho(t)\Vert,\epsilon)}
\end{equation}

$h$ is the parameter called kernel bandwidth, the value of $h$ could largely influnce the smoothness and therefore the visualization, original authors of KDEEB offered a heuristic method to pick the values, need not to get optimal since it suffices for iterative process. With simplification, an update of process can be formulated as:

\begin{equation}
D^{i+1}(G) = \{x^{i+1}=x+\varepsilon\frac{\nabla\rho}{\Vert\nabla\rho\Vert}|x\in D^i(G)\}
\end{equation}

$D^i$ denotes the graph drawing of time interval $i$, the iterative advection process would be terminated in few iteration and end up with bundling result $B(D(G))$.

\subsubsection{Fourier Transform Process}
In Eqn 3., $\rho$ is the convolution model for generating density map, the result of this process is basically obtained from the convolution of $K$ and $D(G)$, and they are both absolutely integrable function thus are feasible objects for Fourier Transform. Then apply the convolution theorem of Fourier Transform on the operation:


\begin{displaymath}
\mathcal{F}[f*g] = \mathcal{F}[f] \cdot \mathcal{F}[g]
\end{displaymath}
\begin{equation}
\rho(x) =  D(G)*\mathcal{F} = \mathcal{F}^{-1}[\mathcal{F}[D(G)]\cdot\mathcal{F}[K]](x)
\end{equation}

$\mathcal{F}^{-1}$ denotes the inverse Fourier Transform and $\cdot$ symbol means the multiplication of two signal function. The convolution theorem shows that a Fourier transform of convolution of two function are equal to the product of their Fourier transform respectively


\section{Algorithm}
The algorithm we proposed is based on the method answered by Hurter, \emph{et al.}\cite{Smooth} The work introduced a way to bundle streaming graphs data that formulated at Eqn 1. with KDEEB, the choice of KDEEB is due to several reasons, and the most deciding one is its computational flexibility. Unlike other methods that would take re-computation once the graphs evolve, the way KDEEB calculates edge advection steps allow it free from re-computation when graph updates. FFTEB has similar process of calculating edge advection steps with KDEEB and thus intuitively suitable for dealing streaming graphs. Our algorithm is illustrated below:


\begin{algorithm}
\caption{Animated-FFTEB}\label{euclid}
\SetKwInOut{KwIn}{Input}
\SetKwInOut{KwOut}{Output}
\SetKwInOut{Initialization}{Initialization}
\KwIn{Vertex-Edge Graph $G = (V,E)$}
\KwOut{Streaming Bundled Graph}
\State $t \gets 0$ \\
\For{the moments edges still exist}{
$E_{l} \gets \{e\in E| t_{end}(e)\cap [t,t+\Delta t] \neq \phi\}$ \\
$\mathcal{F}(D(G)) \gets \{FFT(x)|x\in V\}$ \\
\For{$e \in E_l$}{
$\rho (e) \gets \mathcal{F}^{-1}[\mathcal{F}(D(G))\cdot \mathcal{F}(K)](e)$ 
} \\
\For{$e \in E|t_{end}(e) \in [t-\delta t,t]$}{
\textit{$\backslash\backslash$ Edge Relaxtion} \\
}
\For{$e \in E_l$}{
$e \gets e + \epsilon\frac{\nabla\rho}{\Vert\nabla\rho\Vert}$ \\
\textit{$\backslash\backslash$ Edge Rendering}
}
$t \gets t + \delta t$
}
\end{algorithm}

What this algorithm do is basically make the dynamic graph keep bundling while time-window sliding. First we initialize the edges that exist in the time-span $[t,t+\Delta t]$, the setting of parameter $\Delta t$ should consider if there is sufficient time for underlying edge bundling algorithm, which is FFTEB in our case, to achieve satisfying visual, and it usually takes several iterations, the advantage of this execution is 



\section{Experiment}


\section{Conclusion}


\bibliographystyle{IEEEtran}
\bibliography{MyRef}


\end{document}

